% !TEX encoding = UTF-8 Unicode
\documentclass[a4paper,12pt]{article}
\usepackage{polski}
\usepackage[utf8]{inputenc}
\usepackage{graphicx}
\usepackage{float}
\usepackage{rotating}
\usepackage{caption}
\usepackage{subcaption}
\usepackage{pdfpages}
\usepackage{ucs}
\usepackage{amsmath}
\usepackage{amsfonts}
\usepackage{amssymb}
\usepackage{ucs}
\usepackage{epstopdf}
\usepackage{epsfig}


\begin{document}
% Strona tytułowa
\begin{titlepage}
\begin{center}
\vspace*{1cm}
{ \Large \textbf{ Założenia projektowe }}\\[1cm]

%{ \Large \textsc{\\ robota mobilnego} }\\[2cm]

{ \Large Projekt specjalnościowy ARR \\
\textbf{Stanowisko do badania systemu sensorycznego cybernetycznej dłoni}}\\
Semestr letni 2016/2017\\[2cm]

\Large{
\textbf{Zespół:}}\\
\large {Artur Błażejewski, 200541 (A. B. odpowiedzialny)\\
Dawid Chechelski, 197002(D. C.)\\
Paweł Jachimowski, 200355 (P. J.)\\
Krzysztof Kwieciński, 200418 (K. K.)\\
Witold Lipieta, 200415 (W. L.)}

\vfill
\Large
Politechnika Wrocławska\\
\large
Wydział Elektroniki\\
Automatyka i Robotyka\\
ARR
\end{center}
\end{titlepage}

% Treść
	\section{Założenia projektowe}
		\begin{enumerate}
			\item Analiza stanowiska badawczego:
				\begin{itemize}
					\item Konstrukcja mechaniczna i elektroniczna ręki~(D.C., P.J.)~22.03
					\item Działanie czujników i sposób komunikacji z mikrokontrolerem~(W.L.)~22.03
					\item Dokumentacja
				\end{itemize}
			\item Zapoznanie się z dokumentacją i kodem obecnego programu do wizualizacji danych.~(K.K.)~22.03
			\item Złożenie zamówienia na zakup brakujących elementów układu. (A.B.)~5.04
			\item Montaż czujników na palcach mechanicznej ręki.~(A.B.)~5.04
			\item Zaprojektowanie interfejsu pośredniczącego między czujnikami a aplikacją komputerową:
				\begin{itemize}
					\item Wybór modułu z mikrokontrolerem (W.L.)~29.03
					\item Opracowanie sposobu komunikacji i akwizycji danych (P.J.)~12.04
					\item Przesyłanie danych do komputera (P.J.)~19.04
				\end{itemize}
			\item Rozwój lub stworzenie programu do wizualizacji danych z czujników.~(K.K.)~26.04
			\item Stworzenie raportu.~17.05
	\end{enumerate}
	
	\section{Stan projektu na dzień 09.05.2017}
		\subsection{Część mechaniczna konstrukcji}
		(montaż czujników)
		\subsection{Część elektroniczna}
			\subsubsection{Płytka i połączenia czujników}
			\subsubsection{Połączenie z komputerem}
		\subsection{Software części elektronicznej}
			\subsubsection{Program na nucleo}
			\subsubsection{Przesyłanie danych do komputera}
		\subsection{Software - wizualizacji}
	
	
	
	
\end{document}
